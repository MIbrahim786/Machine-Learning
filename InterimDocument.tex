\documentclass{report}

\usepackage{a4}
\usepackage[hypertex]{hyperref}

\begin{document}

\title{Exploring Machine Learning:\\
  The ID3 algorithm}

\author{Mohammed Ibrahim\\
 Computer Science Department\\
  College of Science, Swansea University\\
  Swansea, SA2 8PP, UK
}

\maketitle

\tableofcontents

\chapter{File Format}
\label{sec:fileformat}

\section{File Format}
\label{sec:file}

\section{Identifiers}
\label{sec:ide}

According to \cite{Roberts2000CompleteJava2Certification}(Chapter 1, page 6): An identifier is a name used by a programmer to variable, method, package, class, interfaces or label. Keywords and reserved words can't be used as  identifiers. It must begin with a letter, a dollar sign, or an underscore; identifiers are case sensitive. Identifiers are tokens (also called symbols) which name as language entities. Each variable has a name by which it is identified in the program.


\section{Separators}
\label{sec:sep}

The JSeparator class provides a horizontal or vertical dividing line or empty space. This type of class commonly used for creation of Menus and tool bars  in Java program.

To refer from \cite{JavaProceduralSyntax}(Separators in Java): Separators help us to define the structure of a program. Below there are some separators which we use in Java program.
\begin{itemize}

\item Parentheses( ) encloses the arguments in method definitions and calling. It also adjust the precedence in the arithmetic expressions.
\item Braces { } defines the block of code and automatically initializes arrays.
\item Square brackets [ ] is used to declares the array types and dereference the array values.
\item Semicolon ; terminates the statements in the program.
\item Coma , separates the successive identifiers in variable declarations; 
\item Dot . is very essential in programming language, it selects a field or method from an object and it separates package names from sub-package and class names.
\item Colon : is used after loops label.

\end{itemize}





\bibliographystyle{plain}
\bibliography{Bibliography}



\end{document}