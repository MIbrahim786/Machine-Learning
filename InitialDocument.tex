% Created 20/12/2011 by Oliver Kullmann (Swansea).

\documentclass{report}

\usepackage{a4}
\usepackage[hypertex]{hyperref}
\usepackage{graphicx}


\begin{document}

\title{Exploring Machine Learning:\\
  The ID3 algorithm}

\author{Mohammed Ibrahim\\
 Computer Science Department\\
  College of Science, Swansea University\\
  Swansea, SA2 8PP, UK
}

\maketitle
\begin{abstract}
  
 This project consists of two software programs, one of which will take a training set as input and produce an output of decision tree. The second stage of this project is to create the second program which will take both training set and the decision tree output from the first program. Finally the resulting output of the second program will produce statistics on the comparison of the values of the target attribute as given by the decision tree and the data table.
  
  
\end{abstract}

\tableofcontents


\chapter{General overview}
\label{cha:genoverview}


\section{Introduction}
\label{sec:int}

This project is based on machine learning. Machine learning is the area of study which enables computers to work together without being explicitly programmed. It describes the technique or methods that involves making the machine to learn and behave based on training data given and past experience to improve its performance. Machine learning computes the partial table to the full table.
There has been a tremendous increase in the field of machine learning over the past decade, not only providing us self-driving cars, practical speech recognition,  effective web search but also enabling us to continually improve the understanding of the human genome. According to \cite{Alpaydin2010MachineLearning}(page 3, Chapter 1):
\begin{quote}
  Machine learning can be described as a technique or method that involves making the machine to learn and behave based on training data given and past experience to improve its performance.
\end{quote}

In this project we are going to use the techniques of ID3 algorithm to implement a unique algorithm. ID3 algorithm deals with the generation of decision tree. As a project scope we are going to build a Java application to test these algorithm. To illustrate the operation of ID3 we are representing the training set.
The training set has been given from the book \cite{Mitchell1997MachineLearning}(page 59, chapter 3):
\begin{quote}
  Machine learning also covers concepts of artificial intelligence (AI) and these techniques are widely used in every field. Face detection, text recognition, strategy games, web searching application are among few to mention that we see in day-to-day life. Machine learning concepts are applied into many other fields such as mathematics, biology and statistics. It mainly focuses on the prediction, based on known properties learned from the training data where as data mining focuses on the discovery of unknown properties on the data.
  The main objective of machine learning is to program computers to use the pattern data or past experience to solve a given problem.
\end{quote}


\subsection{Machine learning}
\label{sec:machinelearn}

Machine learning involves selecting the most appropriate hypothesis from an extensive database of hypotheses. The selected hypothesis ascertains the learner's knowledge and collected data in the best and possible manner.To referred  from \cite{Mitchell1997MachineLearning}(page 14, Chapter 1):

Machine learning addresses the question of how to build computer programs that evolve with every task they perform. Machine learning algorithms can be utilised in various application domains. They are especially in these areas: 
\begin{itemize}
\item Poorly understood difficult domains where humans might not have the knowledge needed to develop effective algorithms;
\item Domains where large databases containing valuable implicit regularities that can be discovered automatically;
\end{itemize}

\begin{quote}
Machine learning exploits concepts from a variety of disciplines such as: Information theory, computational complexity, artificial intelligence, probability and statistics, psychology and neurobiology, control theory, and philosophy.
\end{quote}

According to \cite{Mitchell1997MachineLearning}(page 59,Chapter 3): Training set : In machine learning training sets are generally called as tables, where each row in the table represents a single training example. 

To illustrate the operation of ID3, we consider the learning task represented by the example training set given in Figure \ref{fig:trainingplaytennis}. We have four attributes Outlook, Temperature, Humidity and Wind. The target attribute is called PlayTennis, which has values boolean value \texttt{yes} or \texttt{no} for different Saturday mornings, is to be predicted based on other attributes of the morning in question.
\begin{figure}[h]
  \centering
  \begin{tabular}{|c|l|l|l|l|l|l|}
    \hline
    Day & Outlook & Temperature & Humidity & Wind & PlayTennis\\
    \hline
    1 & sunny & hot & high & weak & no
    \\\hline
    2 & sunny & hot & high & strong & no
    \\\hline
    3 & overcast & hot & high & weak & yes
    \\\hline
    4 & rain & mild & high & weak & yes
    \\\hline
    5 & rain & cool & normal & weak & yes
    \\\hline
    6 & rain & cool & normal & strong & no
    \\\hline
    7 & overcast & cool & normal & strong & yes
    \\\hline
    8 & sunny & mild & high & weak & no
    \\\hline
    9 & sunny & cool & normal & weak & yes
    \\\hline
    10 & rain & mild & normal & weak & yes
    \\\hline
    11 & sunny & mild & normal & strong & yes
    \\\hline
    12 & overcast & mild & high & strong & yes
    \\\hline
    13 & overcast & hot & normal & weak & yes
    \\\hline
    14 & rain & mild & high & strong & no
    \\\hline
  \end{tabular}
  \caption{Training set for \texttt{PlayTennis} example}
  \label{fig:trainingplaytennis}
\end{figure}

From the given training set we see:
\begin{itemize}
\item Outlook variable has 3 values (``sunny, overcast, rain'').
\item Temperature variable has 3 values (``hot, mild, cool'').
\item Humidity variable has 2 values (``high, normal'').
\item Wind variable has 2 values (``weak, strong'').
\end{itemize}
The total number of combinations the table could have is $3 \cdot 3 \cdot 2 \cdot 2 = 36$ possible inputs for the function. We having only $14$ possible values in the table, while 22 values are missing. To find out the missing values we compute a decision tree. Once we know a decision tree it is easy to complete the table. We can summarise:
\begin{center}
  Machine learning is\\
  the completion of a partial table (the given training set)\\
  to a full table (using some bias).
\end{center}

In order to see better the requirements, we outline a simple fictional use-case:
\begin{enumerate}
\item There is one player, John, playing (sometimes) tennis Saturday morning in the tennis centre.
\item The manager David of the tennis centre wants to predict when John might be coming, so that resources can be managed more effectively.
\item After having talking to John, David thinks that the four attributes (``input variables'') \texttt{Outlook}, \texttt{Temperature}, \texttt{Humidity} and \texttt{Wind} are enough to predict the target attribute (``output variable'') \texttt{PlayTennis}.
\item David then collects data over a year, resulting in 14 different weeks, represented in Figure \ref{fig:trainingplaytennis}.
\item Now David asks the machine learning centre MI to predict from this partial data, the ``training set'', the missing data.
\item MI believes that there should be a relatively simple model, and so finding a small decision tree, using the ID3 algorithm, should be a good choice.
\item The algorithm is run, computing a decision tree $T$.
\item Now, once having any decision tree $T$ for the data (there are many!), if David has a query, that is, David knows the values for the four attributes for the next Saturday, this fictional row of the ``complete table'' is entered into the algorithm for evaluating $T$, and the value of \texttt{PlayTennis} is obtained. David reserves accordingly for John a space or not.
\item But before embarking on a permanent contract with MI for answering such queries (every query costs a lot of money), David wants to evaluate the quality of the prediction (that is, the quality of $T$). Since he is not dumb, he withheld from the year of observation $10$ further observations ($10$ more rows with values for inputs and output). Before now entering the contract, these $10$ queries with given outputs (the ``test set'') are fed into $T$ (resp.\ the evaluation-program), and the number of correct predictions is output. David will then decide whether the performance seems sufficient.
\end{enumerate}


\subsection{Decision trees}
\label{sec:dectree}

To cite from \cite{Mitchell1997MachineLearning}(page 59, chapter 3): A ``decision tree'' is a special method to represent the partial training set in a complete form. Based on the bias ``good representations are short'', the heuristics construct a ``short'' decision tree. A basic heuristics is captured by the ID3 algorithm. According to \cite{Mitchell1997MachineLearning}(page 55, chapter 3) The ID3 algorithms learns decision trees by constructing them top down, each instance attribute is evaluated using a statistical test to determine how well it alone classifies the training examples. The best attribute is selected and used as the test at the root node of the tree. A descendant of the root node is then created for each possible value of this attribute , and the training examples are sorted to the appropriate descendant node. The entire process is then repeated using the training examples associated with each descendant node to select the best attribute to test at that point in the tree. 

Explanation of decision trees: To refer from \cite{Mitchell1997MachineLearning}(page 52, chapter 3): The decision tree defines the unique path from the root of the tree to some leaf node, which provides the classification of the instance. Once we compute decision tree it is easy to complete the training set table. 

It returns ``positive'' or ``negative'' value based on that decision can be taken for a tested variable. At a lower level decision trees can be also be represented in the form of if-then rules that can be easily understand.


According to \cite{Mitchell1997MachineLearning}(page 52,chapter 3): Decision tree classifies in the form of tree structure where each branch node represents a choice between a number of alternatives, and each leaf node represents a decision. Decision tree mainly classify data using attributes and it consists of decision nodes and leaf nodes. The tree has number of branches representing with the tested attribute values. Leaf node attribute generate uniform result and it doesn't require any additional classification testing. A leaf node indicates the value of target attribute where as a decision node states some test which can be carried out on single attribute-value, with one branch and sub-tree for each possible outcome of the test.

Decision Tree classify instances by sorting them down the tree from the root to some leaf node, which provide the classifications of the instances, Each node in the tree specifies a test of some attribute of the instances and each branch descending.
Decision tree are commonly used for obtaining information for the purpose of decision making. The tree starts with root node then user split each node recursively according to decision tree learning algorithm.
According to \cite{Mitchell1997MachineLearning}(page 52, chapter 3): Decision tree learning is a method for approximating discrete-valued target functions, in which the learned function is represented by a decision tree. Decision tree learning is one of the most widely used and practical methods for inductive inference. The 3 widely used decision tree learning algorithms are: ID3, ASSISTANT and C4.5. Based on research have decided to do an implementation on ID3 algorithm. 


\subsection{ID3}
\label{sec:ID3}

According to \cite{OverviewOfDecisionTrees}(page 1): ID3 algorithm is invented by Ross Quinlan, it used to generate the decision tree. 
To cited from \cite{OverviewOfDecisionTrees}(page 1): ID3 searches through the attributes of the training instances and extracts the attribute that best separates the given examples. If the attribute perfectly classifies the training sets then the ID3 stops; otherwise it recursively operates on the m (where m = number of possible values of an attribute) partitioned subsets to get their "best" attribute. \\

According to the book of \cite{Mitchell1997MachineLearning}(page 56, chapter 3) ID3(Examples, target-attribute, attribute)
\begin{itemize}
\item Examples : are the training examples.
\item target-attribute : is the attribute whose value is to be predicted by the tree.
\item Attributes : is a list of other attributes that may be tested by the learned decision tree.\\
Returns a decision tree that correctly classifies the given examples.
\end{itemize}

\begin{tabbing}
Create a root node for the tree\\
If \= all examples are positive, Return the single-node tree Root, with label = +.\\
If \= all examples are negative, Return the single-node tree Root, with label = -.\\
If \= Attributes is empty, Return the single-node tree Root, with label = most common value of\\ target attribute in Examples\\
Otherwise Begin \\
\> then \=  A = The Attribute that best classifies examples.\\
\> Decision Tree attribute for Root = A.\\
\> For each possible value, vi, of A,\\
\> \> Add a new tree branch below Root, corresponding to the test A = vi.\\
\> \> Let Examples(vi) be the subset of examples that have the value vi for A\\
\> \> If Examples(vi) is empty\\
\> \> Then below this new branch add a leaf node with label = most\\
\> \> common target value in the examples\\
\> \> Else below this new branch add the subtree ID3 \\
\> \> (Examples(vi), target attribute, Attributes – {A})\\

end\\
return Root\\
\end{tabbing}

The fundamental idea of ID3 is to build the decision tree by using top-down greedy search through the given sets to test each attribute at every tree node.
According to \cite{Mitchell1997MachineLearning}(page 60, chapter 3): In case of ID3 algorithm, it takes three set of parameters.
First is an example that represents the training set. Here training set contains both positive and negative samples. Target attribute is the one whose value has to be determined by using decision tree. And third parameter is the list of attributes that will be tested by the decision tree. Attribute selection is an important part of ID3 algorithm. With the attribute selection step, two terms comes into picture: Entropy and Information gain. With the attribute selection process, the algorithm decides which attribute will be appropriate for becoming a node in the tree.
For an instance play ball. In this example, outlook, temperature, humidity, wind, play ball are attributes. Out of this attributes play ball is considered as classifier because depending on the value of play ball (yes or no), the decision will be made whether tennis can be played or not.


\section{Tools}
\label{sec:Tools}

\subsection{Source control management (Git)}
\label{sec:scm}

Source control management (Git):

Git helps users to communicate securely with remote repository at GitHub.com. It is a remote repository hosting provider with which we can share projects.	
According to \cite{Chacon2011ProGit} (page 5, chapter 1): Git is a powerful, fastest, sophisticated distributed version control system this is quickly replacing subversion in open source and corporate programming communities. It is written in C language and it is active from several years. It designed to handle extremely large projects with speed and efficiency,	but just as well suited for small personal repositories; it is especially popular in the open source community, serving as a development platform for projects like the Linux Kernel, Ruby on Rails, WINE or X.org.
Birth of GIT:
According to \cite{Chacon2011ProGit}(page 5, chapter 1): In the year 2002 Linus Benedict Torvalds uses Bit-Keeper for tracking Linux when it gets better, he writes his own Source Control Management, GIT. Later GIT officially used to track Linux and released GIT 1.5.0 version in the year 2007.I will be using 1.7.8 version.
Repository:  A repository is a set or collection of commits, the work which you have done past it shows in an archive and looks like project's working tree in your machine or someone else's. It holds a set of branches and tags, to identify certain commits by name.

The index : Git does not commit changes directly from the working tree into repository. Changes are first registered in the index,it is the way of confirming changes one by one before doing any commit.																										

Working tree :the directory in a file system called working tree which has repository by stating extension .git and it includes all the files and sub directories in that  directory.

Commit : the word "commit" is often used by git, other revision control systems use the words "version".


\subsection{Unit testing (JUnit)}
\label{sec:junit}

According to the book of \cite{Jones1990SoftwareEngineering}(page 392, Chapter 8): Unit testing : It is the process of taking a program module and running it in isolation from the rest of the software by using prepared input and comparing the actual results with the results  predicted by the specifications and design of the module. The main purpose of testing is to find and debug the errors in the program.\\
There are three main reasons for unit testing :
\begin{enumerate}
\item The size of a single module is small enough that we can locate an error fairly easily.
\item Confusing interactions of multiple errors in widely different parts of the software are eliminated.
\item The module is small enough that we can attempt to test it in some demonstrably exhaustive fashion.
\end{enumerate}
To refer from \cite{JUnitTestingUtilityTutorial}(page 1): JUnit is a framework which helps to test, debug and implement the program in Java. It provides a simple way to explicitly test specific areas of a Java program, it is extensible and can be employed to test a hierarchy of program code either singularly or as multiple units.

JUnit helps the idea of first testing then coding, in that it is possible to setup test data for a unit which defines what the expected output is and then code until the tests pass. It is believed by some that this practice of "test a little, code a little, test a little, code a little..." increases programmer productivity and stability of program code whilst reducing programmer stress and the time spent debugging

According to \cite{JUnitTestingUtilityTutorial}(page 1): JUnit helps first testing then coding, in the program it is possible to arrange the test data for a unit which defines what the expected output is and then code until the tests pass.\\
In our project the program will be tested and implemented same time using JUnit testing.  


\subsection{Document writing (Latex)}
\label{sec:latex}

This is the first time that I have used latex to write my project report.Using this programme has been a highly educational experience. 



\subsection{Documentation of source code}
\label{sec:documentsource}

To implement the software we will be using Doxygen and Javadoc tools. 

To refer from \cite{Doxygen}(page 1): Doxygen Documentation : Doxygen is a documentation system which generate multiple programming languages. Example C, C++, Java, Python, PHP. It is a tools which we use to write a software reference documentation. The documentation is written within the code, and is relatively easy to keep up to date.
It helps in three ways: 
\begin{itemize}
\item It can generate an on-line documentation browser in the form of HTML code and off-line reference manual in Latex from a set of documented source files. It also generate output in (Ms-Word) in RTF format, PostScript, hyperlinked PDF, compressed HTML, and Unix man pages. The documentation is extracted directly from the sources, which makes it much easier to keep the documentation consistent with the source code.
\item doxygen can be configured to extract the code structure from undocumented source files. Doing this type is very useful find quick in large source distributions.
\item One can `abuse' doxygen for creating normal documentation.
\end{itemize}
Doxygen is developed under Linux and Mac OS X, it is highly portable. It runs on most other Unix flavours as well. It supports a number of outputs formats where HTML is the most popular one. 

To refer from \cite{Wikipedia_Javadoc}(page 1):  Javadoc : Javadoc is a documentation generator from Sun Microsystems  for generating API documentation in HTML format from Java source code. To refer from \cite{Javadoc}(page 1):  It is a tool that parses the declarations and documentation comments in a set of source files and produces a set of HTML pages describing the classes, interfaces, constructors, methods and field. 

\section{Evaluation and application}
\label{sec:eval}

The main idea of this project is to implement basic ID3 algorithm, the algorithm will be tested using Java application or Java programs. The project has three main phases and will require two programs ``ComputeDecisionTree" and ``ApplyDecisionTree". 


\subsection{Phase 1: Implementation}
\label{sec:phase1}

In the first phase I will develop the basic ID3 algorithm, this will then be implemented into our program ``ComputeDecisionTree". This program will read the data table (the ``training set") from a specific file. The program will execute and will produce two files. The first file will have the decision tree and in addition to a second file containing statistics that correspond to the algorithms implemented. The statistics contained within the second file will 
include information such as the size of the total function, size of complete table, comparison of the size of the training set and size of the decision tree.

In the second phase the program ``ApplyDecisionTree " will read two files. The first file is the decision tree produced from phase 1 and the second file is the data table (the ``training set"). The program will then produce statistics on the comparison of the values of the target attribute (output variable) as given by the decision tree and the data table.


\subsection{Phase 2: Thorough testing, and additional documentation}
\label{sec:phase2}

To refer from \cite{Wikipedia_Unittesting}(page 1): In our project we will be going to apply Unit testing and application testing.
In application the unit testing is a smallest part and it is a method by which individual units of source code are tested to decide if they are fit for use. In object oriented programming a unit is often an entire interface, such as a class, but could be an individual method. This tests are created by programmer during the development process. Application testing deals with tests for the entire application. 

\subsection{Phase 3: Evaluation and application}
\label{sec:phase3}

Important is evaluation. Implementation of our training sets is evaluated, this evaluation will be able to find how often these sets are correct and how often they are not.

A training set of an unseen situation can be placed into our algorithm, which in turn will give an answer. Then we can compare whether the evaluation given is the one we wanted.

To test the algorithm we can check by assuming all answers given by it are false. Then an example can be applied which we have read up upon from literature or case studies. the evaluation given by our algorithm can then be compared with the ones from the source document.




\chapter{Methodology and Requirements Document}
\label{cha:methrecdoc}

\section{Methodology}
\label{sec:meth}

To refer from \cite{Hamlet2001TheEngineeringofSoftware}(page 15, Chapter 1): Waterfall model: The waterfall model is a sequential design process, which has distinct goals for each phase of development. The first use of this model in the year 1950. In waterfall model, the whole process of software development is divided into separate phases. These phases in waterfall model are:
\begin{itemize}
\item Requirement specification phase
\item Software design
\item Implementation 
\item Testing 
\end{itemize}

This project will be implemented using the Waterfall software development model. We will be taking each phase of the model and applying it to each section of this project.  


\section{Functional requirements}
\label{sec:functreq}

This section contains all the general and user functional requirements for two different software.
\begin{enumerate}

\item The ID3 algorithm is required to provide a software testing functionality.

\item The ID3 algorithms is required to provide the user functionality to input training set data through file input.

\item A training set of an unseen situation can be placed into our algorithm to test the data.

\item The software is required to provide the user functionality to input training set file(file format to be specified).

\item Software should be able to read any types of training sets (data) from a file.

\item Software should be able to check the given training sets.

\item Software should be able to compute the decision tree in a machine readable form(not human readable form).

\item The software must ensure that the computed decision tree will be in separate file.

\item User is able to visualise decision tree by providing small set of data(training set) to the software.

\item Users should able to see the statistics in a separate file.

\item In the second software the user should be able to input two separate file for some statistics result.

\item The software is able to read the first file containing (decision tree). 

\item The software is able to read the second file containing data table(training set). 

\item The software should be able to produce statistics on the comparison of the values of the target attribute as given by the decision tree and the data table.

\item The software should be able to answer all types of queries.

\end{enumerate}

\section{Risk management}
\label{sec:riskman}

During the development of this project, I may be encounter many types of risks. To overcome this risk one must first identify them and then plan accordingly.
In order to avoid difficulties I have listed the risks associated with the project and have developed strategies on how to best minimise these risk.
There are two main categories of risk:
\begin{enumerate}

\item \texttt{General Risk:} These are risks not directly associated with the design of the project. General risks are usually universal and can affect anyone. Illness is a general risk as it can reduce productivity. I had fallen ill while working on this project which significantly reduced my productivity. I was granted an extension to make up for the time that was lost due to my illness. I will try to reduce this risk by maintaining a healthy lifestyle.
 
Significant coursework load will reduce the time that I have to spend on this project. However, I will plan out a schedule with comfortable safety margins which will allow me to finish this project on time and to the best of my ability. Planning out the time I spend on this project will ensure that I give each aspect the attention that it needs.

The topic that I have chosen to investigate is new to me so I will require extra time to do preliminary read. Also English is not my first language so reading and writing technical English words will be more time consuming for me.
 
\item \texttt{Technical Risk::} I will implement two different programs: `` ComputeDecisionTree" and
``ApplyDecisionTree". This means that if one programme malfunctions it will affect the function of the other. Also there is a slight risk of the programmes not being compatible. These issues can reduce productivity and delay reaching the end goal. These risks can be minimised by dividing each programme into smaller segments. The functionality and compatibility of the segments will be tested using Junit testing.

\end{enumerate}


\chapter{Specification Document}
\label{cha:specdoc}

The main idea of this project is to implement the basic ID3 algorithm, and the algorithms will be tested using Java application software. The project is split into two types of software given as ``ComputeDecisionTree" and ``ApplyDecisionTree". 

    
\begin{enumerate}


\item The basic ID3 algorithm will be implemented to test the software application.

\item The software will allow users to input the training set file (file format to be specified).
   
\item The software should produce the two different files.
  
\item The first file containing the decision tree.

\item The computed decision tree must not be visible to the users, it is generated in a machine readable form. 
	
\item By giving a small example (training sets), users will be able to visualise the decision tree. 
	
\item The second file will contain the statistics that corresponds to the algorithms implemented.
	
\item The statistics file include the information such as: size of total function, size of complete table, comparison of the size of the training set and the size of decision tree.
	
\item The other peace of software will read two files from the given input.

\item The software reads the first file called ``decision tree file"", which we produced in the first task.

\item The software reads the second file called ``training set"".
	
\item At last the program will then produce statistics on the comparison of the values of the target attribute (output variable) as given by the decision tree and the data table.

\end{enumerate}


\bibliographystyle{plain}

\bibliography{Literature}

\end{document}


