% Created 20/12/2011 by Oliver Kullmann (Swansea).

\documentclass{article}

\usepackage{a4}
\usepackage[hypertex]{hyperref}
\usepackage{graphicx}


\begin{document}

\title{Exploring Machine Learning:\\
  The ID3 algorithm}

\author{Mohammed Ibrahim\\
 Computer Science Department\\
  College of Science, Swansea University\\
  Swansea, SA2 8PP, UK
}

\maketitle
\begin{abstract}
  XXX
\end{abstract}

\tableofcontents


\section{Introduction}
\label{sec:int}

This project is based on machine learning. Machine learning is the area of study which enables computers to work together without being explicitly programmed. It describes the technique or methods that involves making the machine to learn and behave based on training data given and past experience to improve its performance. Machine learning computes the partial table to the full table.
There has been a tremendous increase in the field of machine learning over the past decade, not only providing us self-driving cars, practical speech recognition,  effective web search but also enabling us to continually improve the understanding of the human genome. According to \cite{Alpaydin2010MachineLearning}(page 3, Chapter 1):
\begin{quote}
  Machine learning can be described as a technique or method that involves making the machine to learn and behave based on training data given and past experience to improve its performance.
\end{quote}

In this project we are going to use the techniques of ID3 algorithm to implement a unique algorithm. ID3 algorithm deals with the generation of decision tree. As a project scope we are going to build a Java application to test these algorithm. To illustrate the operation of ID3 we are representing the training set.
The training set has been given from the book \cite{Mitchell1997MachineLearning}(page 59, chapter 3):
\begin{quote}
  Machine learning also covers concepts of artificial intelligence (AI) and these techniques are widely used in every field. Face detection, text recognition, strategy games, web searching application are among few to mention that we see in day-to-day life. Machine learning concepts are applied into many other fields such as mathematics, biology and statistics. It mainly focuses on the prediction, based on known properties learned from the training data where as data mining focuses on the discovery of unknown properties on the data.
  The main objective of machine learning is to program computers to use the pattern data or past experience to solve a given problem.
\end{quote}


\subsection{Machine learning}
\label{sec:machinelearn}

Machine learning involves selecting the most appropriate hypothesis from an extensive database of hypotheses. The selected hypothesis ascertains the learner's knowledge and collected data in the best and possible manner.To referred  from \cite{Mitchell1997MachineLearning}(page 14, Chapter 1):

Machine learning addresses the question of how to build computer programs that evolve with every task they perform. Machine learning algorithms can be utilised in various application domains. They are especially in these areas: 
\begin{itemize}
\item Poorly understood difficult domains where humans might not have the knowledge needed to develop effective algorithms;
\item Domains where large databases containing valuable implicit regularities that can be discovered automatically;
\end{itemize}

\begin{quote}
Machine learning exploits concepts from a variety of disciplines such as: Information theory, computational complexity, artificial intelligence, probability and statistics, psychology and neurobiology, control theory, and philosophy.
\end{quote}

XXX ``partial table'' XXX ``training set'' XXX ``attributes'' XXX ``variables'' XXX input and output variables XXX ``full table'' XXX 3 pages XX

Machine learning is the completion of the partial table (the given training sets) to a full table.

XXX From the given example of training set, the day attributes contains 14 possible values and the missing values are 22. To finding out missing values we compute decision tree, once we compute decision tree it is easy to complete the table. XXX



\subsection{Decision trees}
\label{sec:dectree}

A Decision tree is a representation of a function from a set of discrete attributes to a classification. Decision tree is a special method to represent the partial training set in a complete form. We intend to create the good and shortest decision tree by applying the ID3 algorithm.  The decision tree defines the unique path from the root to some leaf. Once we compute decision tree it is easy to complete the training set. 

Decision tree is a learning technique that is used to test an object and analyse it. It returns positive or negative value based on that decision can be taken for a tested object. At a lower level decision trees can be also be represented in the form of if-then rules that can be easily understand.

According to \cite{Mitchell1997MachineLearning}(page 52,chapter 3): Decision tree classifies in the form of tree structure where each branch node represents a choice between a number of alternatives, and each leaf node represents a decision. Decision tree mainly classify data using attributes and it consists of decision nodes and leaf nodes. The tree has number of branches representing with the tested attribute values. Leaf node attribute generate uniform result and it doesn't require any additional classification testing. A leaf node indicates the value of target attribute where as a decision node states some test which can be carried out on single attribute-value, with one branch and sub-tree for each possible outcome of the test.

Decision Tree classify instances by sorting them down the tree from the root to some leaf node, which provide the classifications of the instances, Each node in the tree specifies a test of some attribute of the instances and each branch descending.
Decision tree are commonly used for obtaining information for the purpose of decision making. The tree starts with root node then user split each node recursively according to decision tree learning algorithm.
According to \cite{Mitchell1997MachineLearning}(page 52, chapter 3): Decision tree learning is a method for approximating discrete-valued target functions, in which the learned function is represented by a decision tree. Decision tree learning is one of the most widely used and practical methods for inductive inference. The 3 widely used decision tree learning algorithms are: ID3, ASSISTANT and C4.5. Based on research have decided to do an implementation on ID3 algorithm. 


\subsection{ID3}
\label{sec:ID3}

ID3 Algorithm:

ID3 algorithm deals with the generation of decision tree, J.Ross Quinlan developed ID3 at the University of Sydney in the year 1983. To cite from \cite{OverviewOfDecisionTrees} XXX what is this ??? XXX
The fundamental idea of ID3 is to build the decision tree by using top-down greedy search through the given sets to test each attribute at every tree node.
According to \cite{Mitchell1997MachineLearning}(page 60, chapter 3): In case of ID3 algorithm, it takes three set of parameters.
(Examples, Target attribute, Attributes)
First is an example that represents the training set. Here training set contains both positive and negative samples. Target attribute is the one whose value has to be determined by using decision tree. And third parameter is the list of attributes that will be tested by the decision tree. Attribute selection is an important part of ID3 algorithm. With the attribute selection step, two terms comes into picture: Entropy and Information gain. With the attribute selection process, the algorithm decides which attribute will be appropriate for becoming a node in the tree.
For an instance play ball. In this example, outlook, temperature, humidity, wind, play ball are attributes. Out of this attributes play ball is considered as classifier because depending on the value of play ball (yes or no), the decision will be made whether tennis can be played or not.


\section{Tools}
\label{sec:Tools}

\subsection{Source control management (Git)}
\label{sec:scm}

Source control management (Git):

Git helps users to communicate securely with remote repository at GitHub.com. It is a remote repository hosting provider with which we can share projects.	
According to \cite{Chacon2011ProGit} (page 5, chapter 1): Git is a powerful, fastest, sophisticated distributed version control system this is quickly replacing subversion in open source and corporate programming communities. It is written in C language and it is active from several years. It designed to handle extremely large projects with speed and efficiency,	but just as well suited for small personal repositories; it is especially popular in the open source community, serving as a development platform for projects like the Linux Kernel, Ruby on Rails, WINE or X.org.
Birth of GIT:
According to \cite{Chacon2011ProGit}(page 5, chapter 1): In the year 2002 Linus Benedict Torvalds uses Bit-Keeper for tracking Linux when it gets better, he writes his own Source Control Management, GIT. Later GIT officially used to track Linux and released GIT 1.5.0 version in the year 2007.I will be using 1.7.8 version.
Repository:  A repository is a set or collection of commits, the work which you have done past it shows in an archive and looks like project's working tree in your machine or someone else's. It holds a set of branches and tags, to identify certain commits by name.

The index : Git does not commit changes directly from the working tree into repository. Changes are first registered in the index,it is the way of confirming changes one by one before doing any commit.																										

Working tree :the directory in a file system called working tree which has repository by stating extension .git and it includes all the files and sub directories in that  directory.

Commit : the word "commit" is often used by git, other revision control systems use the words "version".


\subsection{Unit testing (JUnit)}
\label{sec:junit}

-XXX sources? XXX

Junit is a tool for project testing and debugging created by Kent beck and Erich Gamma.
JUnit is an open source java testing framework which uses annotations to identify methods that are tests. It also called as tool for project testing and debugging. JUnit assumes that all test methods will be executed in an arbitrary order. Consequently tests should not depend on other tests.
JUnit is a member of the xUnit testing frame work family and now the de facto standard testing framework for java development. JUnit is an Application Program Interface(API) that facilitate developers to easily create Java test cases.
Eclipse IDE supports creating test cases and running test suites, so it is easy to test java application.
I will be creating my test case with the help of Junit tool in the Eclipse IDE.

Features of JUnit Testing Framework
\begin{itemize}

\item It provides a comprehensive assertion facility to test expected results.
\item Test fixtures for sharing common test data.
\item Test suites for easily organizing and running tests.
\item Graphical and textual test runners.

Once the application will be completed, it will be tested with help of JUnit tool XXX in the eclipse IDE  JUnit has nothing to do with any IDE XXX.


\end{itemize}

\subsection{Document writing (Latex)}
\label{sec:latex}

\subsection{Documentation of source code}
\label{sec:documentsource}

\begin{itemize}
\item XXX \textit{Doxygen} \url{http://www.stack.nl/~dimitri/doxygen/} XXX to be used
\item XXX \textit{Javadoc} \url{http://www.oracle.com/technetwork/java/javase/documentation/index-jsp-135444.html}
\end{itemize}





\section{Evaluation and application}
\label{sec:eval}

The main idea of this project is to implement basic ID3 algorithm, the algorithm will be tested using Java application or Java programs. The project has three main phases and will require two programs ``ComputeDecisionTree" and ``ApplyDecisionTree". 


\subsection{Phase 1: Implementation}
\label{sec:phase1}

In the first phase I will develop the basic ID3 algorithm, this will then be implemented into our program ``ComputeDecisionTree". This program will read the data table (the ``training set") from a specific file. The program will execute and will produce two files. The first file will have the decision tree and in addition to a second file containing statistics that correspond to the algorithms implemented. The statistics contained within the second file will 
include information such as the size of the total function, size of complete table, comparison of the size of the training set and size of the decision tree.

In the second phase the program ``ApplyDecisionTree " will read two files. The first file is the decision tree produced from phase 1 and the second file is the data table (the ``training set"). The program will then produce statistics on the comparison of the values of the target attribute (output variable) as given by the decision tree and the data table.

XXX unit-testing XXX


\subsection{Phase 2: Thorough testing, and additional documentation}
\label{sec:phase2}

XXX further unit-tests XXX application-tests XXX

documentation completed (Doxygen) XXX


\subsection{Phase 3: Evaluation and application}
\label{sec:phase3}

Important is evaluation. Implementation of our training sets is evaluated, this evaluation will be able to find how often these sets are correct and how often they are not.

A training set of an unseen situation can be placed into our algorithm, which in turn will give an answer. Then we can compare whether the evaluation given is the one we wanted.

To test the algorithm we can check by assuming all answers given by it are false. Then an example can be applied which we have read up upon from literature or case studies. the evaluation given by our algorithm can then be compared with the ones from the source document.



\section{Methodology and requirements document}
\label{sec:methrecdoc}

\subsection{Methodology}
\label{sec:meth}


\subsection{Functional requirements}

\begin{center}
\begin{tabular}{|r|l|}

\hline
Requirements & Specifications\\[5pt]
\hline

METINFOREQ1 &
The application should be a stand-alone.
\\\hline

METINFOREQ2 &
The training set is created.
\\\hline

METINFOREQ3&
ID3 Algortihm is implemented.
\\\hline

METINFOREQ4&
Decision tree is created.
\\\hline

METINFOREQ5&
Giving different types of training sets to test the algorithm.
\\\hline

\end{tabular}
\end{center}

\subsection{Non-functional requirements}

\begin{tabular}{|r|l|}

\hline
Requirements & Specifications\\[5pt]
\hline
NFR1&
User friendly GUI
\\\hline

NFR2&
Impressive colour schemes in the GUI
\\\hline

NFR3&
Java based
\\\hline

\end{tabular}

\subsection{Risk management}
\label{sec:riskman}




\bibliographystyle{plain}

\bibliography{Literature}

\end{document}


