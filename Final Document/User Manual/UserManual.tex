\documentclass{article}
\usepackage{a4}
\usepackage{hyperref}
\usepackage{graphicx}
\usepackage{listings}
\usepackage{tabularx}
\usepackage{pifont}

\begin{document}

\title{Exploring Machine Learning:\\
  The ID3 algorithm\\
  		User Manual}


\author{Mohammed Ibrahim\\
 Computer Science Department\\
  College of Science, Swansea University\\
  Swansea, SA2 8PP, UK
}

\pagebreak

\maketitle
\pagebreak

\tableofcontents


\abstract{This User Manual details all of the features of the two programs with the screen shots, and briefly explained how to run both the programs. }
\pagebreak


\section{Installaion}
\label{sec:inst}

To run both the programs user must download and install Java. The user can be able to run the program using any number of tools like example, Eclipse, Netbeans, Textpad etc.

\section{Operation}
\label{sec:op}

\subsection{Training set}
\label{sec:tr}

As to check the program how it works, user must need have a data sets. The data sets examples have to follow our specified file format. Below I have given the specified input file format. By having input file format we eliminate the requirement that the program should sort the input attributes in ordered to be processed.
Any input file should follow the following input file format otherwise the program generates the error to the user that the file format is invalid and can't be processed.


The input file format rules are:
\begin{enumerate}

\item Comments are ignored in the file by using double slash ($//$) symbol. 
\item Blank line also be ignored in the data file.
\item The attributes and values are separated by using single space.
\item Every file should have end of line.
\item The very first recognizable line of the data file have to be an attributes.
\item The number of attributes is inferred from the number of words in this line.
\item The last word of the attributes is taken as the {\emph target attribute. }
\item In the data file each subsequent line contains the values of attributes for a data point.
\end{enumerate}

\pagebreak

\begin{figure}[hbtp]
\centering
\includegraphics[scale=1]{../../../scr/valid.png}
\caption{Valid training set}
\end{figure}

The above figure shows the valid training set file that includes 4 attributes and a target attributes. There are 14 possible values corresponding to each attributes.

\pagebreak

\subsection{Compile and Run First Program}
\label{sec: comp}

When the data sets are ready, the user can be able input the data in first program. 
Following steps have to be followed to compile and run the programs.
To Compile in any terminal {\bf javac firstProg.java}
To Run the program in terminal  {\bf java firstProg Tennis\_example}

\begin{figure}[hbtp]
\centering
\includegraphics[scale=0.8]{../../../scr/d2.png}
\caption{Decision Tree for the above training set}
\end{figure}

The above figure shows the decision tree in the form of if,else clause for the above tennis example. It also shows the size of the complete table.

\pagebreak


\subsection{Compile and Run Second Program}
\label{sec:crprog}

Second program checks the produced decision tree with the help of validation sets.\\
Validation set does not include the target attribute.\\
User can be able to add more possible values to check the validation set.\\

\begin{figure}[hbtp]
\centering
\includegraphics[scale=0.8]{../../../scr/check.png}
\caption{Validation set}
\end{figure}

The above figure shows the validation set has 16 possible values with out the target attribute. We give this validation set then the program outputs the target attribute.

\pagebreak

\begin{figure}[hbtp]
\centering
\includegraphics[scale=0.8]{../../../scr/secondprog.png}
\caption{Second program Output }
\end{figure}

The above figure shows the predicted attribute with error rate 0. In the validation set we have added two rows which includes possible values. Both the results are `Yes' so the error rate is 0. 
\pagebreak

The below figure shows that the validation file has two extra possible values in which one predicted attribute is given as `no'. 
The program gives the right predicted attributes for both the values and it calculate the percentage of error rate. Error rate is a simple formula to give an idea about the accuracy of the decision tree. It calculates the proportion of examples that have been misclassified by the decision tree.

\begin{figure}[hbtp]
\centering
\includegraphics[scale=0.6]{../../../scr/sc22.png}
\caption{Validation set and Output}
\end{figure}

\pagebreak



















\end{document}